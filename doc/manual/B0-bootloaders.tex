\chapter{Bootloaders}

\section{Keith\&Koep ``e-boot'' (TrizepsVI)}

Four parts

\begin{description}
\item[\texttt{autoboot.bat}] To be copied into the e-boot boot
  partition. Adjust it to local requirements
\begin{lstlisting}[language=sh]
:) bfmt                     # optionally; erases previous boot partition
:) mount nanddisk
:) del autoboot.bat
:) store mmc2 autoboot.bat  # 'mmc'  ... external slot,
                            # 'mmc2' ... on-module slot
\end{lstlisting}
\item[\texttt{boot.bin}] First stage bootloader; to be copied into the first
  (*FAT) partition of the sd-card
\item[\texttt{zImage-kk-trizeps6.bin}] Flat linux kernel; to be copied
  into the first first (*FAT) partition of the sd-card and renamed to
  \texttt{linux.ram}
\item[\texttt{elito-image-kk-trizeps6.tar.bz2}] RootFS; to be
  extracted into an ext4 partition of the sd-card. Extraction must be
  done as root to set correct file ownership.
\end{description}

\section{Keith\&Koep i-PAN4 + Trizeps MX28}

\begin{enumerate}
\item Build complete set of images (normal and rescue images plus the
  resulting stream)
  \begin{lstlisting}[language=sh]
    make all-images
    make image-stream
  \end{lstlisting}
  You will get two files
  \begin{description}
  \item [\texttt{tmp/deploy/images/rescue-zimage-\$PROJECT-image.sb}]
    Freescale SB stream going to be downloaded by the
    \texttt{imx-usbdownloader} tool below
  \item [\texttt{tmp/deploy/images/\$PROJECT-image.stream}] A stream
    containing kernel, bootloader, rootfs and other data; it can be
    transmitted over TCP, HTTP or on SD Card to the rescue system
    which writes theses parts at the right places.  Integrity of the
    hunks can be guaranteed by x.509 signatures or simple hashes.
  \end{description}
\item Configure device to boot from USB
  \begin{description}
  \item[i-PAN4] Remove the SD Card
  \item[Trizeps MX28] tbd; might be not possible withouth hardware
    modifications atm (2013-04-05)
  \end{description}
\item Execute as root
  \begin{lstlisting}[language=sh]
    imx-usbdownloader -f .../tmp/deploy/images/rescue-zImage-kk-ipan4.sb
  \end{lstlisting}

  The \texttt{imx-usbdownloader} tool must be built either manually;
  see a sample
  \href{http://www.sigma-chemnitz.de/elito/sources/imx-usbdownloader-0.0.2-1.fc17x.src.rpm}{RPM}
  for reference.  Alternatively, it can be built with
  \begin{lstlisting}[language=sh]
    ./bitbake imx-usbdownloader
    ./bitbake imx-usbdownloader-native
  \end{lstlisting}
  The results of the first command are to be executed on the target
  platform; the second command requires that the \texttt{libudev}
  development files are installed on the host.
\item Turn on the target device; it will use DHCP to obtain an IP
  address so look into the logfiles of the dhcp server or dump the
  network traffic to find out the address.  Alternatively, the ip
  address will be given out on the DUART serial port (requires a
  special flat cable and ttl/rs232 adapter).
\item Transmit the image stream
  \begin{lstlisting}[language=sh]
    nc $IP 8000 < tmp/deploy/images/\$PROJECT-image.stream
  \end{lstlisting}
% $
  Progress will be printed on stdout; see
  \ref{app:log:mx28_rescue_system} for sample logs.
\item Reset the device
\end{enumerate}

%%% Local Variables:
%%% mode: latex
%%% TeX-master: "main"
%%% End:
