\chapter{Bootloaders}

\section{Keith\&Koep ``arnoldboot'' (TrizepsIV)}

Three kernel image types:

\begin{description}
\item[\texttt{*-sdram.bin}]
\item[\texttt{*-flash.bin}]
\item[\texttt{*-gdb.bin}]
\end{description}

\begin{verbatim}
fuseburn 9
fuselist
\end{verbatim}



\begin{verbatim}
setbaud com1 115200

boot mmc kernel.img
boot mmc rootfs.img

tftp
\end{verbatim}

\section{Toradex Bootloader (Colibri270)}

\paragraph{NOTE:} Content of this section is preliminary and are
extracts from e-mail communication about creation of U-Boot images.
Integration into the buildsystem is missing.  It assumes the Fedora
implementation of the tftp client program.


\subsection{Creating .nb images}

\paragraph{NOTE:} This section describes generation of U-Boot images
which are to be loaded at address 0.


\begin{enumerate}
\item build the ce-bootme tool
\begin{verbatim}
tar xjf ce-bootme.tar.bz2
make
\end{verbatim}
\item create the .nb image from the u-boot.bin image
\begin{verbatim}
ce-data2bin u-boot.bin 0 > u-boot.nb
\end{verbatim}
\end{enumerate}


\subsection{Using .nb images}

\begin{enumerate}
\item build the ce-bootme tool
\begin{verbatim}
tar xjf ce-bootme.tar.bz2
make
\end{verbatim}
\item install the tftp-client package
\item execute it
\begin{verbatim}
./ce-bootme u-boot-SAx.nb COLIBRI\*
\end{verbatim}%
%
\paragraph{NOTE:} program must be started as root because it listens
  on privileged UDP port 980

\item connect to the ffuart interface of the colibri module with 9600 baud

\item turn off the colibri module

\item press SPACE

\item turn on the colibri module; you will see

\begin{verbatim}
Toradex Bootloader 3.6 for Colibri Built Jun 24 2010

Press [SPACE] to enter Bootloader Menu



BootLoader Configuration:

C) Clear Flash Registry
X) Enter CommandPrompt Mode
D) Download image to RAM now
F) Download image to FLASH now
L) Launch existing flash resident image now
\end{verbatim}

\item press F

\item ce-bootme from step 3 will output something like
\begin{verbatim}
ID: COLIBRI0
mode set to octet
Connected to 192.168.2.126 (192.168.2.126), port 980
putting u-boot-SA0.nb to 192.168.2.126:boot.bin [octet]
Sent 145955 bytes in 6.1 seconds [190260 bit/s]
\end{verbatim}

     and the eboot bootloader

\begin{verbatim}
-EbootSendBootmeAndWaitForTftp
FLASH download [ 0x00000000 ==> 0x00023A37 ]

INFO: FlashErase: 0x00000000 to 0x0002FFFF.  Please wait: 100%
INFO: FlashWrite: 0x00000000 to 0x00023A37.  Please wait: 100%
Flash image. Resetting device...
\end{verbatim}

\end{enumerate}


%%% Local Variables:
%%% mode: latex
%%% TeX-master: "main"
%%% End:
