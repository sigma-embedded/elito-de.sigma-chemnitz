\chapter{Build system setup}
\label{chap:setup}

\section{Services}
\label{sec:setup_services}

\subsection{dnsmasq}
\label{sec:setup_services_dnsmasq}

\begin{enumerate}
\item Install the dnsmasq package of your distribution.
\item Edit the /etc/resolv.conf add, before all other namespace definition
\begin{lstlisting}[language=sh]
nameserver 127.0.0.1
\end{lstlisting}
\item Start the dnsmasq server
  \begin{description}
  \item[Red Hat/Fedora/CentOS:] \lstinline[language=sh]{/sbin/service dnsmasq start}
  \item[Ubuntu/Debian:] \lstinline[language=sh]{/etc/init.d/dnsmasq start}
  \end{description}
\end{enumerate}

You can check if you set it up correct, by issueing this command:

\begin{lstlisting}[language=sh]
dig sigma-chemnitz.de
\end{lstlisting}


If the output ends with something like this, everything went fine.
\begin{lstlisting}[language=sh]
;; Query time: 0 msec
;; SERVER: 127.0.0.1#53(127.0.0.1)
\end{lstlisting}

\subsection{DHCP Server}
\label{sec:setup_services_dhcp-server}

\begin{enumerate}
\item Using the dnsmasq package (see \autoref{sec:setup_services_dnsmasq})
\item Edit the /etc/dnsmasq.conf as needed
Here a quick overview of the options with examples
\begin{lstlisting}[language=sh]
# needs to be set for the DHCP server to run
# specifies the ip pool and lease time
# (here: 192.168.0.100 - 192.168.0.254, lease: 12h)
dhcp-range=192.168.0.100,192.168.0.254,12h

# give a host a static ip
# (here: Host with MAC 11:22:33:44:55:66 => 
#   Name: fred, IP: 192.168.0.60, lease: infinite)
dhcp-host=11:22:33:44:55:66,fred,192.168.0.60,infinite

# set the default router, that should be used by the client (here: 1.2.3.4)
dhcp-option=option:router,1.2.3.4

# for more information look into the conf file (/etc/dnsmasq.conf)
\end{lstlisting}
\item Restart the dnsmasq server
  \begin{description}
  \item[Red Hat/Fedora/CentOS:] \lstinline[language=sh]{/sbin/service dnsmasq restart}
  \item[Ubuntu/Debian:] \lstinline[language=sh]{/etc/init.d/dnsmasq restart}
  \end{description}
\end{enumerate}

\subsection{TFTP Server}
\label{sec:setup_services_tftp-server}

\begin{enumerate}
\item Fedora (19)
  \begin{enumerate}
  \item Install the tftp packages of your distribution.
    \begin{lstlisting}[language=sh]
sudo yum install tftp tftp-server
    \end{lstlisting}
  \item Edit the tftp-server config file (/etc/xinet.d/tftp)
    \begin{enumerate}
      \item change the value of \lstinline[language=sh]{disabled} from yes to no
      \item adjust the other options as needed
    \end{enumerate}
  \item Start the tftp-server
    \begin{lstlisting}[language=sh]
systemctl start xinetd.service
systemctl enable xinetd.service
    \end{lstlisting}
  \end{enumerate}
\item Ubuntu (12.04)
  \begin{enumerate}
  \item Install the atftp packages of your distribution.
    \begin{lstlisting}[language=sh]
sudo apt-get install atftpd atftp tftp
    \end{lstlisting}
  \item Edit the atftp-server config file (/etc/default/aftpd)
    \begin{enumerate}
      \item change the value of \lstinline[language=sh]{USE\_INETD} from true to false
      \item adjust the other options as needed
    \end{enumerate}
  \item Start the atftp-server
    \begin{lstlisting}[language=sh]
sudo /etc/init.d/atftpd restart
    \end{lstlisting}
  \end{enumerate}
\end{enumerate}

If you want to create a new directory for tftpd don't forget to set the right permissions.
\begin{lstlisting}[language=sh]
sudo mkdir -p  \$ftpd\_dir
sudo chmod -R 777 \$ftpd\_dir
sudo chown -R nobody \$ftpd\_dir
\end{lstlisting}

You can test if your Setting are correkt bei putting a test file into your tftpd directory and getting it via tftp.
\begin{lstlisting}[language=sh]
cp \$testfile \$ftpd\_dir
tftp $ip <<EOF
get $testfile
EOF
# use your ip above - localhost will eventually not work
# (depending on your setup)
\end{lstlisting}

\subsection{NFS Server}
\label{sec:setup_services_nfs-server}

Using an NFS rootfilesystem speeds up developing because files built
and installed on the developing machine are immediately visible in the
target system.  To use such an NFS rootfilesystem follow these steps:

\begin{enumerate}
\item Enable \texttt{IMAGE\_KEEPROOTFS} in the project configuration
\item Call \texttt{./configure} with an
  \texttt{-\mbox{}-enable-nfs-root=<dir>} option. The ELiTo buildsystem hides
  this call behind a global \texttt{NFSROOT} configuration variable
  and a project dependent subdirectory.  This subdirectory is usually
  set in the \texttt{build-setup} configuration file in the project
  directory; e.g.
\begin{lstlisting}[language=make]
CONFIGURE_OPTIONS += \
	--enable-machine=toradex-colibri320 \
	--enable-nfs-root=${NFSROOT}/my-project \
	...
\end{lstlisting}%$
  \texttt{NFSROOT} is either given on the make commandline
\begin{lstlisting}[language=bash]
$ make configure M=$project NFSROOT=/srv/elito/.sysroots
\end{lstlisting}
or set in global configuration files\footnote{\texttt{.config} files
  with domain and hostname suffixes are read in the toplevel
  directory}.
\item Build the project. This will populate the directory above with
  the final filesystem.
\end{enumerate}

The previous steps can resp. should be done under a non-privileged
user account.  Installation and configuration of the NFS server
requires superuser permissions.

\begin{enumerate}[resume]
\item Install the NFS server package of your distribution which must
  provide an NFSv3 server. NFSv4 is \textbf{not} supported.
\item Start the NFS server;

  \begin{description}
  \item[Red Hat/Fedora/CentOS:] \lstinline[language=sh]{/sbin/service nfs start}
  \item[Ubuntu/Debian:] \lstinline[language=sh]{/etc/init.d/nfs-* start}
  \end{description}
\item Add to \texttt{/etc/exports} an entry like
\begin{verbatim}
/srv/elito/.sysroots   *(ro,async,no_root_squash)
\end{verbatim} and reload it by
\begin{lstlisting}[language=sh]
# exportfs -ra
\end{lstlisting}

  Although later only the \texttt{/srv/elito/.sysroots/my-project}
  directory is mounted, it is recommended to export the whole
  \texttt{NFSROOT} hierarchy.

\end{enumerate}

The installation can be checked by
\begin{lstlisting}[language=sh]
# mount -tnfs -oro,v3,nolock,tcp \
  `hostname`:/srv/elito/.sysroots/my-project /mnt
$ ls -l /mnt
\end{lstlisting}%$
  which will mount the rootfilesytem on the development machine in a
  way how it will be done later on the device.

  \begin{enumerate}[resume]
  \item Configure the target device to use this location.  The exact
    procedure depends on the used bootloader but basically,
\begin{verbatim}
root=nfs ro nfsroot=<ip>:/srv/elito/.sysroots/my-project,v3,tcp,nolock
\end{verbatim}
    parameters must be added to the kernel cmdline:

    \begin{description}
    \item[direct kernel configuration:] Modify
      \texttt{CONFIG\_CMDLINE} in the kernel configuration.  Please
      note that this should be done only for development purposes; the
      final image should not need a special bootloader configuration
      of the kernel cmdline but should boot with the configured
      \texttt{CONFIG\_CMDLINE}.
    \item[U-Boot:] TBD
    \item[Keith\,\&\,Koep arnoldboot:] The Keith\,\&\,Koep bootloader
      does not allow modification of the kernel cmdline so it must be
      set directly in the kernel configuration as stated above.
    \item[mobm320:] TBD
      See \href{https://www.cvg.de/people/ensc/pxa320/BoardBringUp.html}{Colibri320 Board Bringup Guide} for instructions.
    \end{description}
  \end{enumerate}


%%% Local Variables:
%%% mode: latex
%%% TeX-master: "main"
%%% End:
