\chapter{\label{app:dirtree}Directory trees}

\section{Main directory tree}
\label{sec:dirtree-main}

\dirtree{%
.1 /srv/elito/\DTcomment{git repository}.
.2 COLIBRI270/.
.3 .tmp.
.3 tmp/.
.4 deploy/.
.5 ipk/.
.6 all/.
.7 \dots{}.
.6 armv5te/.
.7 \dots{}.
.7 bzip2\_1.0.5-r2\_armv5te.ipk.
.7 \dots{}.
.6 \dots{}.
.5 images/.
.3 Makefile.
.3 \dots{}.
.2 COLIBRI320/.
.3 \dots{}.
.2 \dots{}.
.2 de.sigma-chemnitz/\DTcomment{git submodule}.
.2 org.openembedded/\DTcomment{git submodule}.
.2 workspace/.
.3 kernel.git\DTcomment{bare git repository}.
.3 u-boot.git\DTcomment{bare git repository}.
.2 Makefile.
.2 project.conf.
}\vspace{1em}

In detail, there is:
\begin{itemize}
\item the \texttt{/srv/elito} toplevel directory with configuration
  data and a makefile with some helper targets
\item the \texttt{org.openembedded} directory with a clone of the
  \href{http://www.openembedded.org}{OpenEmbedded} environment plus
  some ELiTo specific changes.
\item the \texttt{de.sigma-chemnitz} directory with the core ELiTo
  data
\item one or more project directories which contain the setup for your
  platform and which will hold the area where packages will be built
\item directories under \texttt{workspace/} with clones of (big)
  foreign projects which need usually local modifications. The Linux
  kernel and the U-Boot bootloader are examples of such projects.
\end{itemize}

\section{NFS exportable sysroots}

\dirtree{%
.1 /srv/sysroots/.
.2 colibri270-evalboard/.
.3 bin/.
.3 etc/.
.3 lib/.
.3 \dots{}.
}

\section{Source cache}

\dirtree{%
.1 /srv/elito/.cache.
.2 setuptools-0.6c9-py2.6.egg.
.2 sources/.
.3 cvs/.
.3 git/.
.4 sources.redhat.com.git.binutils.git/.
.4 \dots{}.
.3 svn/.
.3 \dots{}.
.3 zlib-1.2.5.tar.bz2.
.3 zlib-1.2.5.tar.md5.
}

%%% Local Variables: 
%%% mode: latex
%%% TeX-master: "main"
%%% End: 
