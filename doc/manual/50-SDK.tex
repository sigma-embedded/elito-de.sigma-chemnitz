\chapter{SDK}
\label{chap:sdk}

These steps will guide you through the creation of the SDK. Its usage and its
integration into Eclipse.

\section{build}
\label{sec:sdk_build}

\begin{enumerate}
\item Switch into your project directory
  \begin{lstlisting}[language=bash]
    cd $project
  \end{lstlisting}%$
\item Build the SDK
  \begin{lstlisting}[language=bash]
    make sdk
  \end{lstlisting}
\item The SDK-Installer is now generated (under: tmp/deploy/sdk/) and can be
  copied (to other systems) and deployed. Just execute the oecore*.sh and follow
  the instructions on the screen.

  See \autoref{app:log:sdk-install} for a sample log.

\end{enumerate}

\section{usage}
\label{sec:sdk_usage}
To use the SDK. You only have to load the conf file, that the installer
generated. It sets all the enviromental variables required.

\begin{lstlisting}[language=bash]
  source $sdkdir/environment-setup-*
\end{lstlisting}

\section{Testing with chroot + Qemu}
\label{sec:sdk_chroot}

To use chroot on the sdk. You need static linked qemu binarys.

You could
\begin{description}
  \item[a)] build them yourself (with ./configure --static)
\end{description}
or
\begin{description}
  \item[b)] obtain the latest deb pkg for static linked qemu binarys and extract
    them from there.
\end{description}

\begin{lstlisting}[language=bash]
  cd /tmp
  deb_mir=http://ftp.us.debian.org/debian/pool
  wget ${deb_mir}/main/q/qemu/qemu-user-static_1.6.0+dfsg-2_amd64.deb
  ar -x qemu-user-static_1.6.0+dfsg-2_amd64.deb
  unxz data.tar.xz --stdout | tar x
  ls usr/bin # static qemu binarys
\end{lstlisting}

Copy the qemu binarys into your sdk and then you should be able to use chroot.
\begin{lstlisting}[language=bash]
  cp qemu-arm-static $sdkdir/sysroot/*linux-gnueabi/usr/bin
  chroot $sdkdir/sysroot/*linux-gnueabi /bin/sh
\end{lstlisting}

\section{Integrate into Eclipse}
\label{sec:sdk_eclipse}

To use the SDK in Eclipse you need to have Eclipse CDT (C/C++ Development Tools)
installed and running. It is also assumed, the you have sucessfully set up the
chroot enviroment (see \autoref{sec:sdk_chroot} for details).

To configure Eclipse to use the sdk, follow these steps.

\begin{enumerate}
\item Open the properties for the project
\item Select C/C++ Build > Settings (on the left side)
\item Select in "Configuration" > "[ All Configurations ]"
\item Set these configurations in the "Tool Setting" tab
(for \$sdkdir use your real path)
\begin{enumerate}
  \item Cross Settings:
\begin{lstlisting}[language=bash]
Prefix: arm-linux-gnueabi-
Path: $sdkdir
\end{lstlisting}
  \item Cross GCC Compiler
\begin{lstlisting}[language=bash]
Command: gcc
\end{lstlisting}
  \item Cross GCC Compiler > Miscellaneous (use full path)
\begin{lstlisting}[language=bash]
Other flags: -c --sysroot=$sdkdir/sysroot/*linux-gnueabi
\end{lstlisting}
  \item Cross GCC Linker
\begin{lstlisting}[language=bash]
Command: gcc
\end{lstlisting}
  \item Cross GCC Linker > Miscellaneous (again, use full path)
\begin{lstlisting}[language=bash]
Linker flags: --sysroot=$sdkdir/sysroots/*linux-gnueabi
\end{lstlisting}
\end{enumerate}
If you are unsure, what to enter - Please take a look at the enviromental* file,
provided by the sdk.
\end{enumerate}

Building your Application should now work, but running the App will fail.
To test run your application you have to add a bash script to you project.
This will the be run using the Run > External Tools menu.

Here's an example of such an script. You have to adjust at least the first 3
lines to use it in your project.

\begin{lstlisting}[language=bash]
export ARM_SDK_PATH=$sdkdir/sysroots/armv5te-linux-gnueabi
export BUILD_BIN_FILES=Debug/HelloWorld
export PGM=HelloWorld

cp $BUILD_BIN_FILES $ARM_SDK_PATH/usr/bin
cat >$ARM_SDK_PATH/doit <<EOF
echo $1
/usr/bin/$PGM
read -rsp $'Press any key to continue...\n' -n1 key
EOF
xterm -e sudo chroot $ARM_SDK_PATH /bin/sh /doit.sh
\begin{lstlisting}[language=bash]
\end{lstlisting}

Now make this script executable by opening the files properties.
Click left on "Ressource" and tick right the permission for execute.

Now tell eclipse, that this script is an external tool.

\begin{enumerate}
\item Open the "External Tools Configurations" (Run > External Tools > ...)
\item Select "Program" on the left side and right-click on it and select "New"
\item Set these configurations in the new created program
\begin{lstlisting}[language=bash]
Name: << Select a name for the external tool menu
Location: << Browse Workspace > Select the shell script
Working directory: << Browse Workspace > Select the workspace
\end{lstlisting}
\end{enumerate}

If everything is correct you should now be able to run your programm by clicking
in the "Run" menu an select your script from the "External Tools" menu.

%%% Local Variables:
%%% mode: latex
%%% TeX-master: "main"
%%% End:
