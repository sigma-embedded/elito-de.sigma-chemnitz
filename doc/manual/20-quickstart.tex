\chapter{Quickstart}
\label{chap:quickstart}

These steps will download the base build environment from the ELiTo
server and will populate directory trees as shown in
\autoref{app:dirtree}. Results are a filesystem exportable by NFS,
various binary images (e.g. for kernel, bootloader and rootfs) and
packages which can be installed manually on the system.

\section{Development host preparations}

Although OpenEmbedded is almost self-hosting, some software is
required for bootstrapping and download tasks:

\begin{itemize}
\item tar, gzip, bzip2, cpio, grep, xz, lzop
\item make, gcc, g++
\item socat, curl, wget, gawk, chrpath
\item git, cvs, svn
\item pkg-config, autoconf, automake, libtool, ccache
\item gnutls development files
\item help2man, diffstat, texi2html, texinfo
\item python-beautifulsoup4, python-inotify
\item python-libxml2 bindings
\item proxy tool for determining local proxy setup (recommended)
\end{itemize}

\subsection*{Ubuntu 10.04 + 12.04.3 LTS}
\begin{lstlisting}[language={}]
sudo apt-get install aptitude
sudo aptitude install \
   tar gzip bzip2 cpio sed xz-utils grep gawk chrpath \
   make gcc g++ pkg-config \
   socat git-core cvs subversion autoconf libtool ccache wget \
   libgnutls-dev help2man diffstat texi2html texinfo \
   python-libxml2 libproxy-tools bzip2 cpio \
   python-pyinotify python-bs4 lzop curl
\end{lstlisting}


\section{ELiTo metadata download}
\label{subsec:quickstart_elito-download}

\begin{enumerate}
\item Verify that requirements are met; especially make sure that
  network access is possible.
\item Create a directory with sufficiently free space (25\,GB) and make
  it writable for your user account.  This step might require
  \texttt{root} permissions.  Due to performance reasons it is not
  recommended to have this directory on an NFS share.

  The following steps will call this place ``toplevel directory'' and
  assume that it is located at \texttt{/srv/elito/bsp}.  The exact
  name and location does not matter (except that it should not be
  excessively long to prevent \texttt{ENAMETOOLONG} errors) and you
  might want to apply a custom naming scheme
  (e.g. \texttt{/srv/elito/bsp-2013.0} or
  \texttt{/srv/elito/myproduct-v1.0}) to allow multiple parallelly
  used branches.
\item\label{step:quickstart_elito-registration} After your
  registration\footnote{see \url{http://elito.sigma-chemnitz.de}} for
  ELiTo you will receive a message with an X.509 certificate and a
  project name (which is used in
  \autoref{step:quickstart_elito-setup-call}).  Install the X.509
  certificate in your home directory as \texttt{\$HOME/.elito.pem}
\item Download the setup
  tool\footnote{\url{http://www.sigma-chemnitz.de/dl/elito/sources/elito-setup}}
  and store it at a temporary place (e.g. as
  \texttt{/tmp/elito-setup}).  This file is basically a
  self-extracting tarball which contains an installation script and
  the sources of an SSL tunnel program.
\item\label{step:quickstart_elito-setup-call} Execute
\begin{lstlisting}[language=bash]
$ bash /tmp/elito-setup /srv/elito/bsp <project> <branch>
\end{lstlisting}%$
  with \texttt{<project>} and \texttt{<branch>} being told in
  \autoref{step:quickstart_elito-registration}. See
  \autoref{app:log:quickstart_elito-setup} for a sample log.

  This command will take some time because large git repositories
  (kernel\footnote{kernel repository has a size of $\sim$500\,MiB} and
  OpenEmbedded\footnote{OpenEmbedded repository has a size of
    $\sim$150\,MiB} clones) will be created by downloading their data
  from the internet.

  \paragraph{Note:} Warnings or error messages are not always fatal for
  this command.  There will be tried fast mirrors first which might
  fail sometimes.
\end{enumerate}

\section{Project build}
\label{subsec:quickstart_project-build}

\begin{enumerate}[resume]
\item Choose and create directories for
  \begin{itemize}
  \item the NFS exportable rootfilesystem (e.g. \texttt{/srv/elito/.sysroots})
  \item downloaded sources (e.g. \texttt{/srv/elito/.cache}).
  \end{itemize}

  These directories can be shared between different projects.

\item Go into the toplevel directory and configure\footnote{the
    \texttt{NFSROOT} and \texttt{CACHEROOT} variables can be put into
    \texttt{.config} files in the toplevel directory; see
    \autoref{sec:XXX}} the project % TODO
\begin{lstlisting}[language=bash]
cd /srv/elito/bsp
make M=$project configure \
  NFSROOT=/srv/elito/.sysroots CACHEROOT=/srv/elito/.cache
make M=$project init
\end{lstlisting}

  See \autoref{app:log:quickstart_elito-configure} for a sample log.

\item (optional) Fetch all sources
\begin{lstlisting}[language=bash]
cd /srv/elito/bsp/$project
./bitbake elito-image -c fetchall
\end{lstlisting}%$

  The normal build process in the next step will download sources
  automatically but the \texttt{fetchall} task allows to detect and
  workaround broken URLs or temporarily unavailable servers.  When a
  broken URL will be reported, the file can be searched in the
  internet and placed into the \texttt{\$CACHEROOT}.
\item (optional) Build an existing project
\begin{lstlisting}[language=bash]
cd /srv/elito/bsp
make M=$project build
\end{lstlisting}%$

\end{enumerate}

\section{Creating a new project}
\label{sec:quickstart_new-project}

\begin{enumerate}
\item Create a project directory
  \begin{lstlisting}[language=bash]
mkdir $new_project
  \end{lstlisting}%$
\item Create a \texttt{build-setup} file in the new project
  directory. There must be set the \texttt{CONFIGURE\_OPTIONS}
  variable which is used to pass several options to
  \texttt{configure}.  See sample projects for a template.
  \begin{lstlisting}[language=bash]
cat > $new_project/build-setup <<EOF
CONFIGURE_OPTIONS += \
	--enable-machine=<machine> \
        --enable-nfs-root=${NFSROOT}/<project> \
        --with-project-name=<project-name>
EOF
  \end{lstlisting}
\item Build the project from the toplevel directory
  \begin{lstlisting}[language=bash]
cd /srv/elito/bsp
make M=<path-to-new-project> build
  \end{lstlisting}
\end{enumerate}

%%% Local Variables:
%%% mode: latex
%%% TeX-master: "main"
%%% End:
